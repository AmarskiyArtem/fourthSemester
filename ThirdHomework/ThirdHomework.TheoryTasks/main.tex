\documentclass{report}

\usepackage{ucs}
\usepackage[utf8x]{inputenc}
\usepackage[russian]{babel}
\usepackage{amsmath, amssymb, scalerel, amsfonts, mathtools}
\setcounter{secnumdepth}{-1}

\begin{document}
\section{Задание 1}
\begin{flalign*}
  & ((\lambda a.(\lambda b.b\:b)\:(\lambda b.b\:b))\:b)\:((\lambda c.(c\:b))\:(\lambda a.a)) \xrightarrow{} \\
  & ((\lambda b.b\:b)\:(\lambda b.b\:b))\:((\lambda c.(c\:b))\:(\lambda a.a)) \xrightarrow{} \\
  & ((\lambda b.b\:b)\:(\lambda b.b\:b))\:((\lambda c.(c\:b))\:(\lambda a.a)) \xrightarrow{} \\
\end{flalign*}
Преобразования "зациклились" По теореме Карри, если бы у терма была нормальная форма, то последовательным сокращением левого редекса мы бы к ней пришли.
\section{Задание 2}
\begin{flalign*}
    & S\:K\:K = (\lambda x\:y\:z.x\:z\:(y\:z))\:(\lambda x\:y.x)\:(\lambda x\:y.x) \xrightarrow{} \\
    & (\lambda y\:z.(\lambda x\:y.x)\:z\:(y\:z))\:(\lambda x\:y.x) \xrightarrow{} \\
    & \lambda z.((\lambda x\:y.x)\:z\:((\lambda x\:y.x)\:z)) \xrightarrow{} \\
    & \lambda z.((\lambda x\:y.x)\:z\:(\lambda y.z)) \xrightarrow{} \\
    & \lambda z.((\lambda y.z)\:(\lambda y.z)) \xrightarrow{} \\
    & \lambda z.z\:=\:I \\
\end{flalign*}
\end{document}
